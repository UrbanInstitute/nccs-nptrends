% Options for packages loaded elsewhere
\PassOptionsToPackage{unicode}{hyperref}
\PassOptionsToPackage{hyphens}{url}
\PassOptionsToPackage{dvipsnames,svgnames,x11names}{xcolor}
%
\documentclass[
  letterpaper,
  DIV=11,
  numbers=noendperiod]{scrreprt}

\usepackage{amsmath,amssymb}
\usepackage{iftex}
\ifPDFTeX
  \usepackage[T1]{fontenc}
  \usepackage[utf8]{inputenc}
  \usepackage{textcomp} % provide euro and other symbols
\else % if luatex or xetex
  \usepackage{unicode-math}
  \defaultfontfeatures{Scale=MatchLowercase}
  \defaultfontfeatures[\rmfamily]{Ligatures=TeX,Scale=1}
\fi
\usepackage{lmodern}
\ifPDFTeX\else  
    % xetex/luatex font selection
\fi
% Use upquote if available, for straight quotes in verbatim environments
\IfFileExists{upquote.sty}{\usepackage{upquote}}{}
\IfFileExists{microtype.sty}{% use microtype if available
  \usepackage[]{microtype}
  \UseMicrotypeSet[protrusion]{basicmath} % disable protrusion for tt fonts
}{}
\makeatletter
\@ifundefined{KOMAClassName}{% if non-KOMA class
  \IfFileExists{parskip.sty}{%
    \usepackage{parskip}
  }{% else
    \setlength{\parindent}{0pt}
    \setlength{\parskip}{6pt plus 2pt minus 1pt}}
}{% if KOMA class
  \KOMAoptions{parskip=half}}
\makeatother
\usepackage{xcolor}
\setlength{\emergencystretch}{3em} % prevent overfull lines
\setcounter{secnumdepth}{5}
% Make \paragraph and \subparagraph free-standing
\ifx\paragraph\undefined\else
  \let\oldparagraph\paragraph
  \renewcommand{\paragraph}[1]{\oldparagraph{#1}\mbox{}}
\fi
\ifx\subparagraph\undefined\else
  \let\oldsubparagraph\subparagraph
  \renewcommand{\subparagraph}[1]{\oldsubparagraph{#1}\mbox{}}
\fi

\usepackage{color}
\usepackage{fancyvrb}
\newcommand{\VerbBar}{|}
\newcommand{\VERB}{\Verb[commandchars=\\\{\}]}
\DefineVerbatimEnvironment{Highlighting}{Verbatim}{commandchars=\\\{\}}
% Add ',fontsize=\small' for more characters per line
\usepackage{framed}
\definecolor{shadecolor}{RGB}{241,243,245}
\newenvironment{Shaded}{\begin{snugshade}}{\end{snugshade}}
\newcommand{\AlertTok}[1]{\textcolor[rgb]{0.68,0.00,0.00}{#1}}
\newcommand{\AnnotationTok}[1]{\textcolor[rgb]{0.37,0.37,0.37}{#1}}
\newcommand{\AttributeTok}[1]{\textcolor[rgb]{0.40,0.45,0.13}{#1}}
\newcommand{\BaseNTok}[1]{\textcolor[rgb]{0.68,0.00,0.00}{#1}}
\newcommand{\BuiltInTok}[1]{\textcolor[rgb]{0.00,0.23,0.31}{#1}}
\newcommand{\CharTok}[1]{\textcolor[rgb]{0.13,0.47,0.30}{#1}}
\newcommand{\CommentTok}[1]{\textcolor[rgb]{0.37,0.37,0.37}{#1}}
\newcommand{\CommentVarTok}[1]{\textcolor[rgb]{0.37,0.37,0.37}{\textit{#1}}}
\newcommand{\ConstantTok}[1]{\textcolor[rgb]{0.56,0.35,0.01}{#1}}
\newcommand{\ControlFlowTok}[1]{\textcolor[rgb]{0.00,0.23,0.31}{#1}}
\newcommand{\DataTypeTok}[1]{\textcolor[rgb]{0.68,0.00,0.00}{#1}}
\newcommand{\DecValTok}[1]{\textcolor[rgb]{0.68,0.00,0.00}{#1}}
\newcommand{\DocumentationTok}[1]{\textcolor[rgb]{0.37,0.37,0.37}{\textit{#1}}}
\newcommand{\ErrorTok}[1]{\textcolor[rgb]{0.68,0.00,0.00}{#1}}
\newcommand{\ExtensionTok}[1]{\textcolor[rgb]{0.00,0.23,0.31}{#1}}
\newcommand{\FloatTok}[1]{\textcolor[rgb]{0.68,0.00,0.00}{#1}}
\newcommand{\FunctionTok}[1]{\textcolor[rgb]{0.28,0.35,0.67}{#1}}
\newcommand{\ImportTok}[1]{\textcolor[rgb]{0.00,0.46,0.62}{#1}}
\newcommand{\InformationTok}[1]{\textcolor[rgb]{0.37,0.37,0.37}{#1}}
\newcommand{\KeywordTok}[1]{\textcolor[rgb]{0.00,0.23,0.31}{#1}}
\newcommand{\NormalTok}[1]{\textcolor[rgb]{0.00,0.23,0.31}{#1}}
\newcommand{\OperatorTok}[1]{\textcolor[rgb]{0.37,0.37,0.37}{#1}}
\newcommand{\OtherTok}[1]{\textcolor[rgb]{0.00,0.23,0.31}{#1}}
\newcommand{\PreprocessorTok}[1]{\textcolor[rgb]{0.68,0.00,0.00}{#1}}
\newcommand{\RegionMarkerTok}[1]{\textcolor[rgb]{0.00,0.23,0.31}{#1}}
\newcommand{\SpecialCharTok}[1]{\textcolor[rgb]{0.37,0.37,0.37}{#1}}
\newcommand{\SpecialStringTok}[1]{\textcolor[rgb]{0.13,0.47,0.30}{#1}}
\newcommand{\StringTok}[1]{\textcolor[rgb]{0.13,0.47,0.30}{#1}}
\newcommand{\VariableTok}[1]{\textcolor[rgb]{0.07,0.07,0.07}{#1}}
\newcommand{\VerbatimStringTok}[1]{\textcolor[rgb]{0.13,0.47,0.30}{#1}}
\newcommand{\WarningTok}[1]{\textcolor[rgb]{0.37,0.37,0.37}{\textit{#1}}}

\providecommand{\tightlist}{%
  \setlength{\itemsep}{0pt}\setlength{\parskip}{0pt}}\usepackage{longtable,booktabs,array}
\usepackage{calc} % for calculating minipage widths
% Correct order of tables after \paragraph or \subparagraph
\usepackage{etoolbox}
\makeatletter
\patchcmd\longtable{\par}{\if@noskipsec\mbox{}\fi\par}{}{}
\makeatother
% Allow footnotes in longtable head/foot
\IfFileExists{footnotehyper.sty}{\usepackage{footnotehyper}}{\usepackage{footnote}}
\makesavenoteenv{longtable}
\usepackage{graphicx}
\makeatletter
\def\maxwidth{\ifdim\Gin@nat@width>\linewidth\linewidth\else\Gin@nat@width\fi}
\def\maxheight{\ifdim\Gin@nat@height>\textheight\textheight\else\Gin@nat@height\fi}
\makeatother
% Scale images if necessary, so that they will not overflow the page
% margins by default, and it is still possible to overwrite the defaults
% using explicit options in \includegraphics[width, height, ...]{}
\setkeys{Gin}{width=\maxwidth,height=\maxheight,keepaspectratio}
% Set default figure placement to htbp
\makeatletter
\def\fps@figure{htbp}
\makeatother
\newlength{\cslhangindent}
\setlength{\cslhangindent}{1.5em}
\newlength{\csllabelwidth}
\setlength{\csllabelwidth}{3em}
\newlength{\cslentryspacingunit} % times entry-spacing
\setlength{\cslentryspacingunit}{\parskip}
\newenvironment{CSLReferences}[2] % #1 hanging-ident, #2 entry spacing
 {% don't indent paragraphs
  \setlength{\parindent}{0pt}
  % turn on hanging indent if param 1 is 1
  \ifodd #1
  \let\oldpar\par
  \def\par{\hangindent=\cslhangindent\oldpar}
  \fi
  % set entry spacing
  \setlength{\parskip}{#2\cslentryspacingunit}
 }%
 {}
\usepackage{calc}
\newcommand{\CSLBlock}[1]{#1\hfill\break}
\newcommand{\CSLLeftMargin}[1]{\parbox[t]{\csllabelwidth}{#1}}
\newcommand{\CSLRightInline}[1]{\parbox[t]{\linewidth - \csllabelwidth}{#1}\break}
\newcommand{\CSLIndent}[1]{\hspace{\cslhangindent}#1}

\KOMAoption{captions}{tableheading}
\makeatletter
\makeatother
\makeatletter
\@ifpackageloaded{bookmark}{}{\usepackage{bookmark}}
\makeatother
\makeatletter
\@ifpackageloaded{caption}{}{\usepackage{caption}}
\AtBeginDocument{%
\ifdefined\contentsname
  \renewcommand*\contentsname{Table of contents}
\else
  \newcommand\contentsname{Table of contents}
\fi
\ifdefined\listfigurename
  \renewcommand*\listfigurename{List of Figures}
\else
  \newcommand\listfigurename{List of Figures}
\fi
\ifdefined\listtablename
  \renewcommand*\listtablename{List of Tables}
\else
  \newcommand\listtablename{List of Tables}
\fi
\ifdefined\figurename
  \renewcommand*\figurename{Figure}
\else
  \newcommand\figurename{Figure}
\fi
\ifdefined\tablename
  \renewcommand*\tablename{Table}
\else
  \newcommand\tablename{Table}
\fi
}
\@ifpackageloaded{float}{}{\usepackage{float}}
\floatstyle{ruled}
\@ifundefined{c@chapter}{\newfloat{codelisting}{h}{lop}}{\newfloat{codelisting}{h}{lop}[chapter]}
\floatname{codelisting}{Listing}
\newcommand*\listoflistings{\listof{codelisting}{List of Listings}}
\makeatother
\makeatletter
\@ifpackageloaded{caption}{}{\usepackage{caption}}
\@ifpackageloaded{subcaption}{}{\usepackage{subcaption}}
\makeatother
\makeatletter
\@ifpackageloaded{tcolorbox}{}{\usepackage[skins,breakable]{tcolorbox}}
\makeatother
\makeatletter
\@ifundefined{shadecolor}{\definecolor{shadecolor}{rgb}{.97, .97, .97}}
\makeatother
\makeatletter
\makeatother
\makeatletter
\makeatother
\ifLuaTeX
  \usepackage{selnolig}  % disable illegal ligatures
\fi
\IfFileExists{bookmark.sty}{\usepackage{bookmark}}{\usepackage{hyperref}}
\IfFileExists{xurl.sty}{\usepackage{xurl}}{} % add URL line breaks if available
\urlstyle{same} % disable monospaced font for URLs
\hypersetup{
  pdftitle={NP Trends Survey Data Preparation Guide},
  pdfauthor={Thiya},
  colorlinks=true,
  linkcolor={blue},
  filecolor={Maroon},
  citecolor={Blue},
  urlcolor={Blue},
  pdfcreator={LaTeX via pandoc}}

\title{NP Trends Survey Data Preparation Guide}
\author{Thiya}
\date{2024-02-28}

\begin{document}
\maketitle
\ifdefined\Shaded\renewenvironment{Shaded}{\begin{tcolorbox}[interior hidden, breakable, frame hidden, sharp corners, borderline west={3pt}{0pt}{shadecolor}, enhanced, boxrule=0pt]}{\end{tcolorbox}}\fi

\renewcommand*\contentsname{Table of contents}
{
\hypersetup{linkcolor=}
\setcounter{tocdepth}{2}
\tableofcontents
}
\bookmarksetup{startatroot}

\hypertarget{preface}{%
\chapter*{Preface}\label{preface}}
\addcontentsline{toc}{chapter}{Preface}

\markboth{Preface}{Preface}

This is a Quarto book.

To learn more about Quarto books visit
\url{https://quarto.org/docs/books}.

\begin{Shaded}
\begin{Highlighting}[]
\DecValTok{1} \SpecialCharTok{+} \DecValTok{1}
\end{Highlighting}
\end{Shaded}

\begin{verbatim}
[1] 2
\end{verbatim}

\bookmarksetup{startatroot}

\hypertarget{introduction}{%
\chapter{Introduction}\label{introduction}}

This is a book created from markdown and executable code.

See Knuth (1984) for additional discussion of literate programming.

\begin{Shaded}
\begin{Highlighting}[]
\DecValTok{1} \SpecialCharTok{+} \DecValTok{1}
\end{Highlighting}
\end{Shaded}

\begin{verbatim}
[1] 2
\end{verbatim}

\bookmarksetup{startatroot}

\hypertarget{summary}{%
\chapter{Summary}\label{summary}}

In summary, this book has no content whatsoever.

\begin{Shaded}
\begin{Highlighting}[]
\DecValTok{1} \SpecialCharTok{+} \DecValTok{1}
\end{Highlighting}
\end{Shaded}

\begin{verbatim}
[1] 2
\end{verbatim}

\bookmarksetup{startatroot}

\hypertarget{references}{%
\chapter*{References}\label{references}}
\addcontentsline{toc}{chapter}{References}

\markboth{References}{References}

\hypertarget{refs}{}
\begin{CSLReferences}{1}{0}
\leavevmode\vadjust pre{\hypertarget{ref-knuth84}{}}%
Knuth, Donald E. 1984. {``Literate Programming.''} \emph{Comput. J.} 27
(2): 97--111. \url{https://doi.org/10.1093/comjnl/27.2.97}.

\end{CSLReferences}

\bookmarksetup{startatroot}

\hypertarget{cleaning-the-survey-data}{%
\chapter{Cleaning the Survey Data}\label{cleaning-the-survey-data}}

In this chapter we process the raw survey data by:

\begin{enumerate}
\def\labelenumi{\arabic{enumi}.}
\tightlist
\item
  Renaming columns
\item
  Recoding response values for quantitative analysis
\item
  Dropping duplicates, incomplete responses and test responses
\end{enumerate}

\hypertarget{renaming-columns}{%
\subsection{Renaming Columns}\label{renaming-columns}}

Columns referencing survey questions are renamed with the help of a data
dictionary to improve readability.

\begin{Shaded}
\begin{Highlighting}[]
\NormalTok{torename }\OtherTok{\textless{}{-}}\NormalTok{ dd }\SpecialCharTok{\%\textgreater{}\%} 
\NormalTok{  dplyr}\SpecialCharTok{::}\FunctionTok{select}\NormalTok{(vname, vname\_raw) }\SpecialCharTok{\%\textgreater{}\%} 
\NormalTok{  tidyr}\SpecialCharTok{::}\FunctionTok{drop\_na}\NormalTok{()}

\NormalTok{survey\_df }\OtherTok{\textless{}{-}}\NormalTok{ raw\_survey\_df }\SpecialCharTok{\%\textgreater{}\%} 
\NormalTok{  dplyr}\SpecialCharTok{::}\FunctionTok{rename\_at}\NormalTok{(}\FunctionTok{vars}\NormalTok{(torename}\SpecialCharTok{$}\NormalTok{vname\_raw), }\SpecialCharTok{\textasciitilde{}}\NormalTok{torename}\SpecialCharTok{$}\NormalTok{vname)}
\end{Highlighting}
\end{Shaded}

\hypertarget{recoding-survey-responses}{%
\subsection{Recoding Survey Responses}\label{recoding-survey-responses}}

The next step is recoding survey responses. Each group of survey
questions comes with its own set of valid inputs that must be recoded
separately. For example, ``N/A'''s are options for some survey questions
and not for others, and some survey questions allow for manual text
inputs.

The below code chunk separates all survey questions into their
respective categories before further separating each category into
numeric, text or NA inputs.

NA inputs here refer to ``Check here if not applicable to your
organization'' columns in the survey, where a ``C'' indicates that the
respondent has checked the N/A box.

\begin{Shaded}
\begin{Highlighting}[]
\CommentTok{\# All questions provided to respondents}
\NormalTok{survey\_qns }\OtherTok{\textless{}{-}} \FunctionTok{names}\NormalTok{(survey\_df)[}\DecValTok{32}\SpecialCharTok{:}\DecValTok{203}\NormalTok{]}

\CommentTok{\# Broken down into specific categories}
\NormalTok{program\_change\_qns }\OtherTok{\textless{}{-}}\NormalTok{ survey\_qns[}\DecValTok{1}\SpecialCharTok{:}\DecValTok{15}\NormalTok{]}
\NormalTok{people\_served\_qns }\OtherTok{\textless{}{-}}\NormalTok{ survey\_qns[}\DecValTok{16}\SpecialCharTok{:}\DecValTok{19}\NormalTok{]}
\NormalTok{demand\_qn }\OtherTok{\textless{}{-}}\NormalTok{ survey\_qns[}\DecValTok{20}\SpecialCharTok{:}\DecValTok{20}\NormalTok{]}

\CommentTok{\# Split into different variable type subgroups}
\NormalTok{staff\_qns }\OtherTok{\textless{}{-}}\NormalTok{ survey\_qns[}\DecValTok{21}\SpecialCharTok{:}\DecValTok{47}\NormalTok{]}
\NormalTok{staff\_qns\_numeric }\OtherTok{\textless{}{-}}\NormalTok{ survey\_qns[}\FunctionTok{c}\NormalTok{(}\DecValTok{21}\SpecialCharTok{:}\DecValTok{27}\NormalTok{, }\DecValTok{29}\SpecialCharTok{:}\DecValTok{36}\NormalTok{, }\DecValTok{38}\NormalTok{)]}
\NormalTok{staff\_qns\_text }\OtherTok{\textless{}{-}}\NormalTok{ survey\_qns[}\FunctionTok{c}\NormalTok{(}\DecValTok{28}\NormalTok{, }\DecValTok{37}\NormalTok{, }\DecValTok{47}\NormalTok{)]}
\NormalTok{staff\_qns\_na }\OtherTok{\textless{}{-}}\NormalTok{ survey\_qns[}\FunctionTok{c}\NormalTok{(}\DecValTok{39}\SpecialCharTok{:}\DecValTok{46}\NormalTok{)]}

\CommentTok{\# Rest of Qns To{-}Do if above are ok}
\NormalTok{importance\_qns }\OtherTok{\textless{}{-}}\NormalTok{ survey\_qns[}\DecValTok{48}\SpecialCharTok{:}\DecValTok{49}\NormalTok{]}
\NormalTok{fundraise\_qns }\OtherTok{\textless{}{-}}\NormalTok{ survey\_qns[}\DecValTok{50}\SpecialCharTok{:}\DecValTok{60}\NormalTok{]}
\NormalTok{majorgift\_qn }\OtherTok{\textless{}{-}}\NormalTok{ survey\_qns[}\DecValTok{61}\SpecialCharTok{:}\DecValTok{61}\NormalTok{]}
\NormalTok{fundraise\_change\_qns }\OtherTok{\textless{}{-}}\NormalTok{ survey\_qns[}\DecValTok{62}\SpecialCharTok{:}\DecValTok{68}\NormalTok{]}
\NormalTok{fundraise\_donor\_qns }\OtherTok{\textless{}{-}}\NormalTok{ survey\_qns[}\DecValTok{69}\SpecialCharTok{:}\DecValTok{70}\NormalTok{]}
\NormalTok{finance\_fundraise\_qns }\OtherTok{\textless{}{-}}\NormalTok{ survey\_qns[}\DecValTok{71}\SpecialCharTok{:}\DecValTok{96}\NormalTok{]}
\NormalTok{finance\_revenue\_qns }\OtherTok{\textless{}{-}}\NormalTok{ survey\_qns[}\DecValTok{97}\SpecialCharTok{:}\DecValTok{105}\NormalTok{]}
\NormalTok{reserve\_qns }\OtherTok{\textless{}{-}}\NormalTok{ survey\_qns[}\DecValTok{110}\SpecialCharTok{:}\DecValTok{111}\NormalTok{]}
\NormalTok{carse\_qns }\OtherTok{\textless{}{-}}\NormalTok{ survey\_qns[}\DecValTok{112}\SpecialCharTok{:}\DecValTok{113}\NormalTok{]}
\NormalTok{finance\_chng\_qns }\OtherTok{\textless{}{-}}\NormalTok{ survey\_qns[}\DecValTok{114}\SpecialCharTok{:}\DecValTok{121}\NormalTok{]}
\NormalTok{leadership\_chng\_qns }\OtherTok{\textless{}{-}}\NormalTok{ survey\_qns[}\DecValTok{122}\SpecialCharTok{:}\DecValTok{132}\NormalTok{]}
\NormalTok{race\_gender\_qns }\OtherTok{\textless{}{-}}\NormalTok{ survey\_qns[}\DecValTok{133}\SpecialCharTok{:}\DecValTok{158}\NormalTok{]}
\NormalTok{extaffairs\_qns }\OtherTok{\textless{}{-}}\NormalTok{ survey\_qns[}\DecValTok{159}\SpecialCharTok{:}\DecValTok{171}\NormalTok{]}
\NormalTok{primary\_cncrn\_qn }\OtherTok{\textless{}{-}}\NormalTok{ survey\_qns[}\DecValTok{172}\SpecialCharTok{:}\DecValTok{172}\NormalTok{]}
\end{Highlighting}
\end{Shaded}

\hypertarget{survey-question-breakdown}{%
\subsubsection{Survey Question
Breakdown}\label{survey-question-breakdown}}

\begin{itemize}
\tightlist
\item
  \textbf{172} general survey questions
\item
  \textbf{15} questions about changes to programs and services
\item
  \textbf{4} questions about the number of people each organization
  serves
\item
  \textbf{1} question about overall program demand
\item
  \textbf{27} questions about staff numbers
\end{itemize}

\begin{Shaded}
\begin{Highlighting}[]
\CommentTok{\# Recode each category}
\NormalTok{survey\_recode\_df }\OtherTok{\textless{}{-}}\NormalTok{ survey\_df }\SpecialCharTok{\%\textgreater{}\%} 
\NormalTok{  dplyr}\SpecialCharTok{::}\FunctionTok{mutate}\NormalTok{(}
\NormalTok{    dplyr}\SpecialCharTok{::}\FunctionTok{across}\NormalTok{(}
\NormalTok{      program\_change\_qns,}
      \SpecialCharTok{\textasciitilde{}}\NormalTok{ dplyr}\SpecialCharTok{::}\FunctionTok{case\_match}\NormalTok{(}
\NormalTok{        ., }
        \StringTok{"Yes"} \SpecialCharTok{\textasciitilde{}} \DecValTok{1}\NormalTok{, }\StringTok{"No"} \SpecialCharTok{\textasciitilde{}} \DecValTok{0}\NormalTok{, }\StringTok{"N/A"} \SpecialCharTok{\textasciitilde{}} \DecValTok{99}\NormalTok{, }
        \StringTok{"Unsure"} \SpecialCharTok{\textasciitilde{}} \DecValTok{98}\NormalTok{, }\StringTok{"{-}99"} \SpecialCharTok{\textasciitilde{}} \SpecialCharTok{{-}}\DecValTok{99}\NormalTok{,}
        \AttributeTok{.default =} \ConstantTok{NA}
\NormalTok{      )}
\NormalTok{    ),}
\NormalTok{    dplyr}\SpecialCharTok{::}\FunctionTok{across}\NormalTok{(}
\NormalTok{      people\_served\_qns,}
      \SpecialCharTok{\textasciitilde{}}\NormalTok{ dplyr}\SpecialCharTok{::}\FunctionTok{case\_match}\NormalTok{(}
\NormalTok{        .,}
        \StringTok{"{-}99"} \SpecialCharTok{\textasciitilde{}} \SpecialCharTok{{-}}\DecValTok{99}\NormalTok{, }\ConstantTok{NA} \SpecialCharTok{\textasciitilde{}} \ConstantTok{NA}\NormalTok{, }\StringTok{"N/A"} \SpecialCharTok{\textasciitilde{}} \DecValTok{1}\NormalTok{,}
        \AttributeTok{.default =} \FunctionTok{as.integer}\NormalTok{(.)}
\NormalTok{      )}
\NormalTok{    ),}
\NormalTok{    dplyr}\SpecialCharTok{::}\FunctionTok{across}\NormalTok{(}
\NormalTok{      demand\_qn,}
      \SpecialCharTok{\textasciitilde{}}\NormalTok{ dplyr}\SpecialCharTok{::}\FunctionTok{case\_match}\NormalTok{(}
\NormalTok{        .,}
        \StringTok{"Increase"} \SpecialCharTok{\textasciitilde{}} \DecValTok{98}\NormalTok{, }\StringTok{"Stay the same"} \SpecialCharTok{\textasciitilde{}} \DecValTok{0}\NormalTok{, }\StringTok{"Decrease"} \SpecialCharTok{\textasciitilde{}} \DecValTok{1}\NormalTok{,}
        \StringTok{"{-}99"} \SpecialCharTok{\textasciitilde{}} \SpecialCharTok{{-}}\DecValTok{99}\NormalTok{,}
        \AttributeTok{.default =} \ConstantTok{NA}
\NormalTok{      )}
\NormalTok{    ),}
\NormalTok{    dplyr}\SpecialCharTok{::}\FunctionTok{across}\NormalTok{(}
\NormalTok{      staff\_qns\_numeric,}
      \SpecialCharTok{\textasciitilde{}}\NormalTok{ dplyr}\SpecialCharTok{::}\FunctionTok{case\_match}\NormalTok{(}
\NormalTok{        .,}
        \StringTok{"{-}99"} \SpecialCharTok{\textasciitilde{}} \SpecialCharTok{{-}}\DecValTok{99}\NormalTok{, }\ConstantTok{NA} \SpecialCharTok{\textasciitilde{}} \ConstantTok{NA}\NormalTok{,}
        \AttributeTok{.default =} \FunctionTok{as.integer}\NormalTok{(.)}
\NormalTok{      )}
\NormalTok{    ),}
\NormalTok{    dplyr}\SpecialCharTok{::}\FunctionTok{across}\NormalTok{(}
\NormalTok{      staff\_qns\_text,}
      \SpecialCharTok{\textasciitilde{}}\NormalTok{ dplyr}\SpecialCharTok{::}\FunctionTok{case\_match}\NormalTok{(}
\NormalTok{        .,}
        \StringTok{"{-}99"} \SpecialCharTok{\textasciitilde{}} \StringTok{"{-}99"}\NormalTok{, }\ConstantTok{NA} \SpecialCharTok{\textasciitilde{}} \ConstantTok{NA}\NormalTok{,}
        \AttributeTok{.default =} \FunctionTok{as.character}\NormalTok{(.)}
\NormalTok{      )}
\NormalTok{    ),}
\NormalTok{    dplyr}\SpecialCharTok{::}\FunctionTok{across}\NormalTok{(}
\NormalTok{      tidyselect}\SpecialCharTok{::}\FunctionTok{all\_of}\NormalTok{(staff\_qns\_na),}
      \SpecialCharTok{\textasciitilde{}}\NormalTok{ dplyr}\SpecialCharTok{::}\FunctionTok{case\_match}\NormalTok{(}
\NormalTok{        .,}
        \StringTok{"{-}99"} \SpecialCharTok{\textasciitilde{}} \SpecialCharTok{{-}}\DecValTok{99}\NormalTok{, }\ConstantTok{NA} \SpecialCharTok{\textasciitilde{}} \ConstantTok{NA}\NormalTok{, }\StringTok{"C"} \SpecialCharTok{\textasciitilde{}} \DecValTok{1}\NormalTok{,}
        \AttributeTok{.default =} \ConstantTok{NA}
\NormalTok{      )}
\NormalTok{    )}
\NormalTok{  )}
\end{Highlighting}
\end{Shaded}

\begin{verbatim}
Warning: There were 7 warnings in `dplyr::mutate()`.
The first warning was:
i In argument: `dplyr::across(...)`.
Caused by warning:
! Using an external vector in selections was deprecated in tidyselect 1.1.0.
i Please use `all_of()` or `any_of()` instead.
  # Was:
  data %>% select(program_change_qns)

  # Now:
  data %>% select(all_of(program_change_qns))

See <https://tidyselect.r-lib.org/reference/faq-external-vector.html>.
i Run `dplyr::last_dplyr_warnings()` to see the 6 remaining warnings.
\end{verbatim}

\begin{Shaded}
\begin{Highlighting}[]
\CommentTok{\# raw\_survey\_df \%\textgreater{}\% }
\CommentTok{\#   dplyr::rowwise() \%\textgreater{}\% }
\CommentTok{\#   dplyr::mutate(response\_rate = sum(dplyr::c\_across(tidyselect::all\_of(survey\_qns)) != {-}99, na.rm = TRUE)) \%\textgreater{}\% }
\CommentTok{\#   dplyr::select(ResponseId, response\_rate) \%\textgreater{}\% }
\CommentTok{\#   dplyr::filter(! ResponseId \%in\% weighted\_survey\_df$ResponseId)}
\end{Highlighting}
\end{Shaded}




\end{document}
